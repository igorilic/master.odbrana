% $Header$

% hyperref to work with cyrillic fonts
\PassOptionsToPackage{unicode}{hyperref}
\documentclass[11pt, serbianc, english]{beamer}

% This file is a solution template for:

% - Talk at a conference/colloquium.
% - Talk length is about 20min.
% - Style is ornate.

% translations
\uselanguage{Serbian}
\languagepath{Serbian}
% \deftranslation[to=Serbian]{Example}{Primjer}
% Copyright 2004 by Till Tantau <tantau@users.sourceforge.net>.
%
% In principle, this file can be redistributed and/or modified under
% the terms of the GNU Public License, version 2.
%
% However, this file is supposed to be a template to be modified
% for your own needs. For this reason, if you use this file as a
% template and not specifically distribute it as part of a another
% package/program, I grant the extra permission to freely copy and
% modify this file as you see fit and even to delete this copyright
% notice.

\mode<presentation>
{
  \usetheme{CambridgeUS}
  \usecolortheme{seahorse}
  \usefonttheme{professionalfonts}
  % or ...

  \setbeamercovered{transparent}
  % or whatever (possibly just delete it)
}


\usepackage[main=serbianc]{babel}
% or whatever

\usepackage[utf8x]{inputenc}
% or whatever

% \usepackage{times}
\usepackage[T2A]{fontenc}
% Or whatever. Note that the encoding and the font should match. If T1
% does not look nice, try deleting the line with the fontenc.


\title[Претраживач - Python] % (optional, use only with long paper titles)
{Примена програмског језика Python у реализацији алгоритама за рангирање веб страница}

% \subtitle
% {Include Only If Paper Has a Subtitle}

\author[Игор Илић] % (optional, use only with lots of authors)
{Игор Илић\inst{}}
%{F.~Author\inst{1} \and S.~Another\inst{2}}
% - Give the names in the same order as the appear in the paper.
% - Use the \inst{?} command only if the authors have different
%   affiliation.

\institute[Универзитет у Београду] % (optional, but mostly needed)
{
  \inst{}%
  Математички факултет\\
  Универзитет у Београду\\
  \includegraphics[scale=0.10]{vecigrb.png}
}
  %\and
 % \inst{2}%
 % Department of Theoretical Philosophy\\
 % University of Elsewhere}
% - Use the \inst command only if there are several affiliations.
% - Keep it simple, no one is interested in your street address.

\date[Мастер 2016] % (optional, should be abbreviation of conference name)
{Одбрана Мастер рада, 2016.}
% - Either use conference name or its abbreviation.
% - Not really informative to the audience, more for people (including
%   yourself) who are reading the slides online

%\subject{Theoretical Computer Science}
% This is only inserted into the PDF information catalog. Can be left
% out.



% If you have a file called "university-logo-filename.xxx", where xxx
% is a graphic format that can be processed by latex or pdflatex,
% resp., then you can add a logo as follows:

\pgfdeclareimage[height=0.5cm]{university-logo}{vecigrb.png}
\logo{\pgfuseimage{university-logo}}



% Delete this, if you do not want the table of contents to pop up at
% the beginning of each subsection:
\AtBeginSubsection[]
{
  \begin{frame}<beamer>{Outline}
    \tableofcontents[currentsection,currentsubsection]
  \end{frame}
}


% If you wish to uncover everything in a step-wise fashion, uncomment
% the following command:

%\beamerdefaultoverlayspecification{<+->}


\begin{document}

\begin{frame}
  \titlepage
\end{frame}

\begin{frame}{Outline}
  \tableofcontents
  % You might wish to add the option [pausesections]
\end{frame}


% Structuring a talk is a difficult task and the following structure
% may not be suitable. Here are some rules that apply for this
% solution:

% - Exactly two or three sections (other than the summary).
% - At *most* three subsections per section.
% - Talk about 30s to 2min per frame. So there should be between about
%   15 and 30 frames, all told.

% - A conference audience is likely to know very little of what you
%   are going to talk about. So *simplify*!
% - In a 20min talk, getting the main ideas across is hard
%   enough. Leave out details, even if it means being less precise than
%   you think necessary.
% - If you omit details that are vital to the proof/implementation,
%   just say so once. Everybody will be happy with that.

\section{Програмски језик Python}

\subsection{Синтакса програмског језика Python}

\begin{frame}{Рачунање и променљиве у Python-у}
    \begin{itemize}
            \item
                Python је моћан калкулатор
            \item
                Променљиве реферишу на вредност која им се додељује
    \end{itemize}
    \begin{example}
        \begin{itemize}
                \item >>> $p = 1$
                \item >>> $q = 2$
                \item >>> $p = q$
                \item >>> print $p$
                \item >>> 2
        \end{itemize}
    \end{example}
\end{frame}

\subsection{Типови података у програмском језику Python}

\subsection{Употреба програмског језика Python}

\section{Претраживање и рангирање веб страница}

\subsection{Креирање и начин рада веб-паука}

\subsection{Рангирање страница}

\section{Закључак}

\end{document}


